\documentclass[12pt]{article}
\usepackage[margin=2cm]{geometry}
\usepackage{titling}
\usepackage{graphicx}
\usepackage{float}
\usepackage[hidelinks]{hyperref}
\usepackage[italian]{babel}
\usepackage{subcaption}

\setlength\parindent{0pt}
\setlength{\parskip}{1em}
\setlength{\droptitle}{-2cm}

\title{Istruzioni d'uso ChocoPrinter}
\author{Università della Svizzera Italiana}
\date{Versione \today}


\begin{document}
\maketitle
\tableofcontents
\newpage

\section{Setup}

	\subsection{Download}\label{sec:download}

		Sono necessari i seguenti software:

		- Repetier-Host, \url{https://www.repetier.com/download-now}

		- Configurazioni per slicer, \url{https://3dprint.elettronicain.it/wp-content/uploads/2015/10/Configurazione-Repetier.zip}

		Per aggiornare il firmware della stampante (opzionale):

		- Arduino IDE 1.0.6, \url{https://www.arduino.cc/en/Main/OldSoftwareReleases}

		- Firmware stampante, \url{TODO}
		
		
	\subsection{Configurazione}
	
		Installare il software Repetier-Host sulla propria macchina ed avviarlo. Cliccare sulla scheda \texttt{Slicer} e in seguito sul pulsante \texttt{Configure}. Cliccare su \texttt{Cancel} per chiudere la procedura guidata e dal menu \texttt{File} scegliere \texttt{Load Config}. Scegliere il file \texttt{Slic3r\_config\_bundle.ini} sotto \texttt{Configurazione-Repetier/3DRAG Choco/Config Slic3r\_CHOCO} dal pacchetto di configurazioni scaricato.\\
		Cliccare sull'ingranaggio accanto al campo \texttt{Print settings}, poi sull'icona di salvataggio accanto al nome del profilo, salvarlo con un nome adatto (p.es. "ChocoPrinter") e confermare con \texttt{Ok}. Chiudere la scheda e ripetere l'operazione con i campi \texttt{Filament} e \texttt{Printer}.\\
		Al termine, chiudere la finestra \texttt{Slic3r} e scegliere i nuovi profili appena creati dal pannello \texttt{Slic3r} di Repetier-Host.
		
		//TODO config stampante?


\section{Stampa}

1) Far partire il SW Repetier-Host

2) Accendere la stampante

3) Collegarsi alla stampante (configurarla se non già fatto)

4) Inserire il cioccolato nella siringa

5) Inserire l'ago nella siringa e poi infilare la siringa nel tondino d'alluminio

6) Inserire la siringa con l'ago nel tondino d'alluminio e posizionare la siringa con il tondino nel cilindro della stampante

7) Se necessario alzare il supporto, cliccando su Retract, per inserire correttamente la siringa

8) Scaldare il cioccolato a 30$^{\circ}$C. All'inizio, se la temperatura supera i 33$^{\circ}$C, spegnere il riscaldamento. Mantenere la temperatura tra i 30 e 33 gradi

9) Attendere che il cioccolato esca dalla siringa

10) Cliccare su Print Panel e posizionare la stampante a "Home All"

11) Cliccare su Object Placement e scegliere l'oggetto (file *.stl) da stampare

12) Modificare la dimensione dell'oggetto se necessario, modificando il numero del campo "Scale"

13) Cliccare su Slicer e selezionare "Slice with Slic3r”

14) Per stampare cliccare su Run


\section{Aggiornamento del firmware}

	Nota: questa procedura richiede una versione 1.0.x di Arduino IDE, che non è compatibile con MacOS da High Sierra 10.13.4 in avanti.

	Scaricare il firmware appropriato (versione 3Drag Choco - LCD xxx, controller verde o nero) dal sito indicato nella sezione \ref{sec:download}.\\
	
	\textbf{IMPORTANTE}: scegliere la versione corretta! Caricare un firmware sbagliato potrebbe rendere la stampante inutilizzabile!\\
	
	Estrarre il pacchetto del firmware ed aprire il file \texttt{Marlin.ino} nella cartella \texttt{Marlin}. Dal menu \texttt{Sketch > Import Library...} scegliere \texttt{Add Library...} e selezionare la cartella	 \texttt{libraries/U8glib} nella cartella scaricata in precedenza.
	Dal menu \texttt{Tools > Board} selezionare \texttt{Arduino Mega 2560 or Mega ADK} e dal menu \texttt{Tools > Serial Port} selezionare la porta COM associata alla stampante. Verificare sulla scheda della stampante che il jumper JPROG sia inserito. Cliccare sul secondo pulsante da sinistra (\texttt{Upload}) e attendere il completamento della procedura. 
	A procedura terminata, scollegare l’alimentazione della stampante o premere il pulsante di reset presente sulla scheda stessa.

\end{document}