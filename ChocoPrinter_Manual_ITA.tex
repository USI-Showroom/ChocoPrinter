\documentclass[12pt]{article}
\usepackage[margin=2cm]{geometry}
\usepackage{titling}
\usepackage{graphicx}
\usepackage{float}
\usepackage[hidelinks]{hyperref}
\usepackage[italian]{babel}
\usepackage{subcaption}

\setlength\parindent{0pt}
\setlength{\parskip}{1em}
\setlength{\droptitle}{-2cm}

\title{Istruzioni d'uso ChocoPrinter}
\author{Università della Svizzera Italiana}
\date{Versione \today}


\begin{document}
\maketitle
\tableofcontents
\newpage

\section{Setup}

	\subsection{Download}\label{sec:download}

		Sono necessari i seguenti software:

		- Repetier-Host, \url{https://www.repetier.com/download-now}

		- Configurazioni per slicer, \url{https://github.com/USI-Showroom/ChocoPrinter/blob/config/Slic3r_config_choco.ini}

		Per aggiornare il firmware della stampante (opzionale):

		- Arduino IDE 1.0.6, \url{https://www.arduino.cc/en/Main/OldSoftwareReleases}

		- Firmware stampante, \url{https://3dprint.elettronicain.it/blog/2012/09/06/software/}
		
		
	\subsection{Configurazione}\label{config}
	
		Installare il software Repetier-Host sulla propria macchina ed avviarlo. Cliccare sulla scheda \texttt{Slicer} e in seguito sul pulsante \texttt{Configure}. Cliccare su \texttt{Cancel} per chiudere la procedura guidata e dal menu \texttt{File} scegliere \texttt{Load Config}. Scegliere il file di configurazione (\texttt{Slic3r\_config\_choco.ini}) scaricato in precedenza.\\
		Cliccare sull'ingranaggio accanto al campo \texttt{Print settings}, poi sull'icona di salvataggio accanto al nome del profilo, salvarlo con il nome desiderato (p.es. "ChocoPrinter") e verificare i seguenti parametri:
		
		- Layers and perimeters $>$ Layer Height: 0.7mm\\
		- Infill $>$ FIll Density: 100\%\\
		- Speed $>$ Speed for print moves: 20 mm/s\\
		
		Se non ci sono errori, confermare con \texttt{Ok}. Chiudere la scheda e ripetere l'operazione con il campo \texttt{Filament}:

		- Filament $>$ Temperature ($^{\circ}$C) $>$ Extruder: 33\\
		- Filament $>$ Temperature ($^{\circ}$C) $>$ Bed: 0\\
		- Cooling $>$ Keep fan always on: On\\
		- Cooling $>$ Fan speed: 100
		
		Stessa cosa per il campo \texttt{Printer}. Al termine, chiudere la finestra \texttt{Slic3r} e scegliere i nuovi profili appena creati dal pannello \texttt{Slic3r} di Repetier-Host.\\
		
		Dalle icone in alto a destra cliccare su \texttt{Printer settings}; assicurarsi di aver scelto la porta corretta e di aver impostato la \texttt{Baud Rate} a 250000.


\section{Stampa}

	\begin{enumerate}
		\item Far partire il software Repetier-Host
	
		\item Cliccare su \texttt{Object Placement} e scegliere l'oggetto (file *.stl) da stampare tramite il pulsante \texttt{Add STL file}
	
		\item Modificare la dimensione dell'oggetto se necessario, modificando il numero del campo \texttt{Scale}

		\item Cliccare su \texttt{Slice with Slic3r}

		\item Accendere la stampante

		\item Collegarsi alla stampante tramite il cavo USB. Assicurarsi di avere configurato il software come descritto al punto \ref{config}

		\item Inserire il cioccolato nella siringa
		
		\item Se necessario alzare il supporto girando la ruota bianca, per poter inserire correttamente la siringa. Qualora il motore dovesse fare resistenza, cliccare su \texttt{Stop Motor} dal pannello \texttt{Print Panel}

		\item Montare l'ago sulla siringa e poi infilare la siringa nel cilindro della stampante, verificando che il tondino di alluminio sia presente in fondo al cilindro

		\item Attendere che il cioccolato esca dalla siringa

		\item Per stampare cliccare su Run
	\end{enumerate}

	Durante la stampa verificare periodicamente che il cioccolato stia uscendo dalla siringa. \textbf{NON} lasciare mai la stampante senza sorveglianza!

\section{Aggiornamento del firmware}

	Nota: questa procedura richiede una versione 1.0.x di Arduino IDE, che non è compatibile con MacOS da High Sierra 10.13.4 in avanti.

	Scaricare il firmware appropriato (versione 3Drag Choco - LCD alfanumerico, controller verde o nero) dal sito indicato nella sezione \ref{sec:download}.\\
	
	\textbf{IMPORTANTE}: scegliere la versione corretta! Caricare un firmware sbagliato potrebbe rendere la stampante inutilizzabile!\\
	
	Estrarre il pacchetto del firmware ed aprire il file \texttt{Marlin.ino} nella cartella \texttt{Marlin}. Dal menu \texttt{Sketch > Import Library...} scegliere \texttt{Add Library...} e selezionare la cartella \texttt{libraries/U8glib} nella cartella scaricata in precedenza.
	Dal menu \texttt{Tools > Board} selezionare \texttt{Arduino Mega 2560 or Mega ADK} e dal menu \texttt{Tools > Serial Port} selezionare la porta COM associata alla stampante. Verificare sulla scheda della stampante che il jumper JPROG sia inserito. Cliccare sul secondo pulsante da sinistra (\texttt{Upload}) e attendere il completamento della procedura. 
	A procedura terminata, scollegare cavo USB e alimentazione della stampante e rimuovere il jumper.

\end{document}